\documentclass[twoside,a4paper,12pt]{book}%,AutoFakeBold,twoside,openright,
% 可用选项有 debug|ebook|hardcopy, amd|pmd|phdplain|phdfancy, nobox, manualSpine, onlyCover, twoSideCover, noBlankPages.
% debug 生成的PDF带框线,方便调试
% ebook 带彩色文字的PDF
% hardcopy 无彩色文字的PDF
% amd 学硕使用
% pmd 专硕使用
% phdplain 博士论文简装
% phdfancy 博士论文精装
% nobox 输出的封面无框线和书脊
% manualSpine 手动输出书脊,需配合\jluManualSpine命令
% onlyCover 仅输出封面页
% twoSideCover 输出双页封面
% noBlankPages 去掉空白页,主要用于上传到图书馆学位论文系统
% 默认为 hardcopy,amd,且nobox=false,manualSpine=false,onlyCover=false,twoSideCover=false, noBlankPages=false
% 双面印刷需在documentclass选项中声明twoside
% 单面印刷需在documentclass选项中声明oneside
% 选项使用举例如下
%\usepackage[phdplain,ebook,twoSideCover,onlyCover]{jluthesis2023}%
% \usepackage[amd,hardcopy,twoSideCover]{jluthesis2023}%
\usepackage[amd,hardcopy,noBlankPages]{jluthesis2023}%
% \usepackage[phdplain,hardcopy,noBlankPages]{jluthesis2023}%

% 用于算法注释
\usepackage{algcompatible}
\renewcommand{\COMMENT}[2][.5\linewidth]{%
  \leavevmode\hfill\makebox[#1][l]{//~#2}}
\algnewcommand\algorithmicto{\textbf{to}}
\algnewcommand\RETURN{\State \textbf{return} }

% 用于排版代码
\usepackage{listings}


\graphicspath{{figures/}}

% \newcommand{\Title}{吉林大学学位论文模板{\jluthesisVersion}示例}
\newcommand{\Title}{中文论文题目}
\newcommand{\Author}{甄继伟} % 作者姓名

\begin{document}
\frontmatter
\sloppy % 解决中英文混排的断行问题,会加入间距,但不会影响断行 ????


% 手动在长标题中利用 \par 输入断行,
\jluCTitle{\Title}

% 竖排的书脊输出时,如果遇到中英混杂的标题会比较麻烦,需要中英分开控制输出
% 但如果仅有中文或英文都比较简单,如
% \jluCSpineTitle{吉林大学学位论文模版示例} % 仅中文的书脊标题
% \jluCSpineTitle{\jluPrintSpine{Sample Typesetting for {\jluthesisVersion}}} % 仅英文的书脊标题
\jluCSpineTitle{\jluPrintSpine{\jluPrintSpineChinese{吉林大学学位论文模版}{\jluthesisVersion}\jluPrintSpineChinese{示例}}} % 中英混杂的书脊标题

\jluSpineHorizontalPosition{1.65} % 书脊的水平位置,修改可使书脊水平平移

% 当在选项中指定手动输出书脊时,可用下列命令手动输出书脊进行微调
% \jluManualSpine{
%   % for oneSideCover
%	\jluPrintVerticallyOneByOne{2.5cm}{-10em}{吉林大学学位论文模板}{0.2in}{}
%	\jluPrintVerticallySentence{2.5cm}{-26.5em}{\rotatebox{-90}{\jluthesisVersion}}
%	\jluPrintVerticallyOneByOne{2.5cm}{-30em}{示例}{0.2in}{}
%	\jluPrintVerticallyOneByOne{2.45cm}{-38em}{ \Author }{0.2in}{\bfseries}
%	\jluPrintVerticallyOneByOne{2.5cm}{-47em}{ 吉林大学 }{0.2in}{\kai}
%   % for twoSideCover    
%	\jluPrintVerticallyOneByOne{23cm}{-10em}{吉林大学学位论文模板}{0.2in}{}
%	\jluPrintVerticallySentence{23cm}{-26.5em}{\rotatebox{-90}{\jluthesisVersion}}
%	\jluPrintVerticallyOneByOne{23cm}{-30em}{示例}{0.2in}{}
%	\jluPrintVerticallyOneByOne{23cm}{-38em}{ \Author }{0.2in}{\bfseries}
%	\jluPrintVerticallyOneByOne{23cm}{-47em}{ 吉林大学 }{0.2in}{\kai}
% } % jluManualSpine
\jluETitle{Title of English Thesis} % 封面的『英文论文题目』
\jluCSubject{计算数学}        % 封面的『专业』或『类别』
\jluCInterest{高性能计算}             % 封面的『研究方向』或『领域(方向)』,博士精装版扉页的『研究方向』
\jluCSubjectScd{计算机应用技术}     % 扉页的『专业名称』或『领域(方向)』
\jluCSubjectTrd{计算机应用技术}     % 学位论文授权说明中的『学科专业』
\jluCDegree{理学硕士}               % 扉页的『学位类别』或『类别』
\jluCCollege{数学学院}
\jluCollegeNumber{10183}
\jluCUniversity{吉林大学}
\jluCAuthor{\Author} % 若为两个字建议中间留一个汉字的空间,如王\hphantom{空}喆
\jluCSupervisor{张\hphantom{空}希} % 若为两个字建议中间留一个汉字的空间,如王\hphantom{空}喆
\jluCSupervisorPost{教授}
\jluClassification{TP391}
\jluSecurityLevel{公开}
\jluCAddress{吉林大学数学学院,130012} 
\jluTelphone{(+86) 1354454XXXX}  
\jluStudentNumber{2023312XXX}

\jluAuthorSignatureOne{ % 原创声明作者签名
	\raisebox{0.5cm}{
		\hspace{3.5cm}
		\includegraphics[width=3cm]{figures/author.png}	
	}
}
\jluAuthorSignatureTwo{ % 授权声明作者签名
	\raisebox{0.5cm}{
		\hspace{3.5cm}
		\includegraphics[width=3cm]{figures/author.png}	
	}
}
\jluSupervisorSignature{ % 授权声明导师签名
	\raisebox{0.5cm}{
		\hspace{3.5cm}
		\includegraphics[width=3cm]{figures/author_tutor.png}
	}
}

% 以下日期若注释掉,对应的日期将留空
\jluDateForCover{2025}{4}
\jluDegreeObtainDate{2025}{6}{21}
\jluDefenseDate{2025}{5}{31}
\jluOriginalStatementDate{2025}{6}{1}
\jluContributionStatementDate{2025}{6}{1}

\jluCommitteeTable{
	%\vspace{5cm}
	\setstretch{1.32}
    \begin{tabular}{l@{\hspace{0.5cm}}l@{\hspace{1.0cm}}l@{\hspace{1.5cm}}r}
    \multicolumn{4}{l}{答辩委员会组成:}\\
            &姓名 & 职称 & 工作单位 \\
        主席&徐宜花& 教授 & XX大学   \\
        委员&李辉京& 教授 & YYYY大学   \\
            &刘世美& 教授 & ZZZZZZZZ大学   \\
            &千允才& 教授 & AA大学   \\
            &张英牧& 教授 & BB大学   \\
            &尹\quad 凡& 研究员 & BB大学   \\
    \end{tabular}
}

\input{part/abstract.tex}

\hypersetup{
	pdfcreator={XeLaTeX \& \jluthesisVersion},   % creator of the document
	pdfproducer={XeLaTeX \& \jluthesisVersion},  % producer of the document
	pdfinfo={
          CreationDate={2020 0401 120000},
          ModDate={2020 0401 120000},
    },
}
\jluMakeCover


\titleformat{\chapter}{\centering\sffamily\mdseries\sanhao}{\CJKchaptername}{1em}{}
\titlespacing{\chapter}{0pt}{8pt}{16pt}

\pagenumbering{Roman} 
\tableofcontents

\cleardoublepage
\pdfbookmark{插图目录}{lof}
\label{lof}
\listoffigures
\clearpage
\pdfbookmark{表格目录}{lot}
\label{lot}
\listoftables

\titleformat{\chapter}{\centering\rmfamily\bfseries\sanhao}{\CJKchaptername}{1em}{}
\titlespacing{\chapter}{0pt}{8pt}{16pt}

\clearpage
\pagenumbering{arabic}


{\xiaosi}
%{\fontsize \fontsize{12.05pt}{14.45pt}\selectfont}
% 清除目录后面空页的页眉和页脚
\clearpage{\pagestyle{empty}\cleardoublepage}

%%% 正文
\mainmatter
\xiaosi                        % 正文使用默认字体,小四,宋体

\input{part/introduction.tex} % 绪论
\chapter{研究现状}
\begin{table}[h]
	\caption{计划进度}  
	\label{tab:schedule}
	\centering
	\scalebox{0.9}{
		\begin{tabular}{lll}
		  \toprule
		  \multicolumn{3}{c}{进度}\\
		  \midrule
		   2007.10~--~2008.05 & & 完成文献综述和开题报告\\
		  
		  2008.05~--~2008.07 & & 完成电渗流多尺度模拟的理论构建和初步模型验证,\\
		                   & & 同时开始写毕业论文\\
		  2008.07~--~2008.10 & & 完成模型验证,并进行多种条件下电渗流(泵)的仿真模拟,\\
		                    & & 实验数据处理,制表,论文书写\\
		  2008.10~--~2008.12 & & 完成所有仿真对象的模拟,并计划完成毕业论文的初稿\\
		  2008.12~--~2009.03 & & 完成论文修改\\
		  2009.03~--~2009.05  & & 准备答辩、答辩\\
		  \bottomrule
		\end{tabular}
	}
\end{table}

\begin{dotsequation}
    \begin{aligned}
		                     a&=b\\
		                     c&=d\\
		                  2x+y&=6\\
		                     x&=4y+\log (y)\\
		\sum_i^{10} i \times x&=y
    \end{aligned}
    \label{eq:log}
\end{dotsequation}

\section{研究现状及挑战}
\label{sec:challenge}
研究现状及挑战见~\autoref{eq:log},多个引用~\cite{Manz1990,Harrison1993,Erickson2003}。

\section{研究内容与论文结构}

研究内容\cite{erdHos1960evolution,konect:socialcomputing,konect}与论文结构\cite{Harrison1993,Li_2020,Erickson2003,Auroux2002}。
 % 研究现状

\chapter{算法}
尝试插入算法(见\autoref{alg:alg1})

\begin{algorithm}[h]
    \setstretch{1.3}
    \textbf{Input:} 两个数 $n$, $k$
    \begin{algorithmic}[1]
        \STATE 初始化 $s$ 为 $0$ \label{algline:init}
        \FOR{$i = 1,2,...,k$}    \COMMENT{\textbf{循环}}
        \label{algline:for} 
        \IF {$i\;\% \;5\neq 0$}
        \STATE $s=s+n+i$
        \ENDIF 
        \ENDFOR \label{algline:endfor}
        \STATE \textbf{return} $s$ \label{algline:return} 
    \end{algorithmic}
    \caption{sum($n$, $k$)}
    \label{alg:alg1}
\end{algorithm} % 算法
\chapter{代码}
尝试插入代码

\definecolor{commentColor}{rgb}{0.0, 0.5, 0.0}
\lstset{ %
    language=Python,                % the language of the code
    basicstyle=\scriptsize\ttfamily,           % the size of the fonts that are used for the code
    numbers=left,                   % where to put the line-numbers
    numberstyle=\tiny\color{gray},  % the style that is used for the line-numbers
    stepnumber=2,                   % the step between two line-numbers. If it's 1, each line 
    % will be numbered
    numbersep=5pt,                  % how far the line-numbers are from the code
    backgroundcolor=\color{white},      % choose the background color. You must add \usepackage{color}
    showspaces=false,               % show spaces adding particular underscores
    showstringspaces=false,         % underline spaces within strings
    showtabs=false,                 % show tabs within strings adding particular underscores
    rulecolor=\color{black},        % if not set, the frame-color may be changed on line-breaks within not-black text (e.g. commens (green here))
    tabsize=2,                      % sets default tabsize to 2 spaces
    captionpos=b,                   % sets the caption-position to bottom
    breaklines=true,                % sets automatic line breaking
    breakatwhitespace=false,        % sets if automatic breaks should only happen at whitespace
                 % show the filename of files included with \lstinputlisting;
    % also try caption instead of title
    keywordstyle=\color{blue},          % keyword style
    commentstyle=\color{commentColor},       % comment style
    stringstyle=\color{mauve},         % string literal style
    escapeinside={(*}{*)},            % if you want to add LaTeX within your code
    morekeywords={*,...}               % if you want to add more keywords to the set
}
\newcommand{\commentMath}[1]{$\color{commentColor}#1$}
\begin{lstlisting}
def func(n, k):
    """
    param n: 起始值 (*\commentMath{n}*)
    param k: 计数 (*\commentMath{k}*)
    """
    s=0
    for i in range(1,k):
        if i % 5 != 0:
            s = s+n+i
    return s
\end{lstlisting} % 代码
% \input{part/experiment.tex} % 实验
% \input{part/conclusion.tex} % 结论
% \input{part/futurework.tex} % 未来工作







\jluReferenceFile{references} % 参考文献文件名
\jluReferenceStyle{gbt7714-numerical} % gbt7714-author-year
\jluPrintReference



\begin{jluSelfIntroduction}
此处放置作者简介
\end{jluSelfIntroduction}



\begin{jluAcknowledgment}
三年时间看似漫长却又一晃而过,回首走过的岁月,我感慨良多。从最初的论文选题、思路梳理到研讨交流、反复修改直至最终完稿,都离不开老师、同学和亲人们的支持和无私帮助,在此我要向他们表达我最诚挚的谢意。

...

求学生涯暂告段落,但求知之路却永无止境。我将倍加珍惜大学生活给予我的珍贵财富,不忘初心,砥砺前行!
\end{jluAcknowledgment}

\appendix
\begin{appendices}
	\chapter{实验所用数据}
	......
	\chapter{其他}
	......
\end{appendices}

\end{document}
