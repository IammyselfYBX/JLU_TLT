\chapter{代码}
尝试插入代码

\definecolor{commentColor}{rgb}{0.0, 0.5, 0.0}
\lstset{ %
    language=Python,                % the language of the code
    basicstyle=\scriptsize\ttfamily,           % the size of the fonts that are used for the code
    numbers=left,                   % where to put the line-numbers
    numberstyle=\tiny\color{gray},  % the style that is used for the line-numbers
    stepnumber=2,                   % the step between two line-numbers. If it's 1, each line 
    % will be numbered
    numbersep=5pt,                  % how far the line-numbers are from the code
    backgroundcolor=\color{white},      % choose the background color. You must add \usepackage{color}
    showspaces=false,               % show spaces adding particular underscores
    showstringspaces=false,         % underline spaces within strings
    showtabs=false,                 % show tabs within strings adding particular underscores
    rulecolor=\color{black},        % if not set, the frame-color may be changed on line-breaks within not-black text (e.g. commens (green here))
    tabsize=2,                      % sets default tabsize to 2 spaces
    captionpos=b,                   % sets the caption-position to bottom
    breaklines=true,                % sets automatic line breaking
    breakatwhitespace=false,        % sets if automatic breaks should only happen at whitespace
                 % show the filename of files included with \lstinputlisting;
    % also try caption instead of title
    keywordstyle=\color{blue},          % keyword style
    commentstyle=\color{commentColor},       % comment style
    stringstyle=\color{mauve},         % string literal style
    escapeinside={(*}{*)},            % if you want to add LaTeX within your code
    morekeywords={*,...}               % if you want to add more keywords to the set
}
\newcommand{\commentMath}[1]{$\color{commentColor}#1$}
\begin{lstlisting}
def func(n, k):
    """
    param n: 起始值 (*\commentMath{n}*)
    param k: 计数 (*\commentMath{k}*)
    """
    s=0
    for i in range(1,k):
        if i % 5 != 0:
            s = s+n+i
    return s
\end{lstlisting}